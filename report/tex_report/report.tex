\documentclass[12pt,a4paper,twoside]{article}

\usepackage{fancyhdr}
\usepackage{lastpage}
\usepackage{a4wide} 
\usepackage{amsmath}
\usepackage{amssymb} 
\usepackage{graphicx}
\usepackage{color}
\usepackage{fancybox}
\usepackage{moreverb}
\usepackage{listings}
\usepackage[utf8]{inputenc}

%\fbox{}
%\shadowbox{}
%\doublebox{}
%\ovalbox{}
%\Ovalbox{}
%\shabox{}


% --- Logo Atos ---
%\makebox[\textwidth][l]{
%\raisebox{-15pt}[0pt][0pt]{
%\hspace{2.5cm}
%\includegraphics[scale=0.1]{Images/logo_atos.eps}
%}
%}
\title{Rapport de projet de fin d'études}
\author{Nom et Prénom Etudiant}
\date{\today}

\pagestyle{headings}

\begin{document}
\lstset{ numbers=left, tabsize=3, frame=single, numberstyle=\ttfamily, basicstyle=\footnotesize} 
\thispagestyle{empty}

\begin{center}
\makebox[\textwidth][l]{
\raisebox{-8pt}[0pt][0pt]{
\includegraphics[scale=0.18]{Images/logo_entreprise}
}
}
\makebox[\textwidth][r]{
\raisebox{0pt}[0pt][0pt]{
\includegraphics[scale=0.2]{Images/logo_ensimag}
}
}
Grenoble INP  -- Ensimag\\
École Nationale Supérieure d'Informatique et de Mathématiques Appliquées\\
\vspace{3cm}
{\LARGE Final year project report}\\
\vspace{1cm}
Effectué chez Nom Entreprise\\
\vspace{2cm}
\shadowbox{
\begin{minipage}{1\textwidth}
\begin{center}
{\Huge Titre Sujet de PFE}\\
\end{center}
\end{minipage}
}\\
\vspace{3cm}
Adrien Gregorj\\
3rd year -- Option MMIS Bio\\
\vspace{3mm}
11 février 2008 -- 01 août 2008\\
\vspace{4cm}
\begin{tabular}{p{10cm}p{10cm}}
{\bf Nom Entreprise}                                            &{\bf Responsable de stage}\\
{\footnotesize Adresse Entreprise}       & ~~~Nom Et Prénom Tuteur Entreprise\\
{\footnotesize BP XX}                                        & {\bf Tuteur de l'école}\\
{\footnotesize 38000 Grenoble Cedex}                          & ~~~Nom et PrénomTuteur Ecole\\
\end{tabular}
\end{center}
\newpage



\section{Context}

\section{Problematic}

Dr Zeynep Yücel has been working for a while on modeling crowd movements. The motivation behind constructing such models are numerous: testing architectural designs for an evacuation process, designing accurate crowd models for movies or video games, etc. Japan being very involved in the robotic industries, these projects guidance was to analyse comportements of pairs of pedestrian (dyads) to potentially develop robot able to walk along side with people in the most natural manner. In order to get real data that could be analysed and from which mathematical models could be derived, cameras and sensors were installed in an underground pedestrian zone. 

Previous work \ref{??} has also shown that using trajectory informations, the composition of a group can be accurately determined. Detection of dyads can be performed relying on data from the sensors that represents the trajectories of the pedestrians.

The goal of this project is to develop a pipeline to automatize the classification process of the dyads gestures. Instead of relying on human referees to decide wether or not a couple of pedestrians are interacting, this decision has to be made by an algorithm.

\section{Pipeline}

The current pipeline that we wish to implement is 

\subsection{Input data}
This section described the initial data that is used when 

\subsection{Pre-processing}
The raw videos from the camera need to be modified in order to be treated. Indeed, the relevant portions that can later be processed by the neural networks need to be selected. The first step is to retrieve all the dyads contained in the videos. This is done by using the annotations files and listing all the pedestrian that interacts with another one.

Camera calibration: in order to be able to map the trajectories to the video to find the appropriate portion to extract, the camera calibration matrix needs to be computed. This is done using OpenCV calibration tool box. 

Bounding box: to compute the actual portion of the screen that contains the walking dyad, a bounding box has been mapped from the real world to the image plan, using the camera matrix computed with the calibration. In the world coordinates, the bounding box is 3 meters wide and 2 meters high so that it surround the two pedestrian. This bounding box is then used as a mask to remove uninteresting information.





\end{document}
